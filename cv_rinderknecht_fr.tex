%%-*-latex-*-

\documentclass[a4paper,11pt]{article}

\usepackage[french]{babel}
\usepackage[T1]{fontenc}    % Required for hyphenation
\usepackage{url}            % Typesets URLs
\usepackage{xspace}         % Adds space after macro expansions if needed
\usepackage{hyphenat}       % Hyphenation of compounded words
\usepackage{mdwlist}        % Tight vertical spacing of list items
\frenchspacing              % Settles the tricky business with \@

\newcommand{\cpp}{\mbox{C \hspace*{-2.5mm} \raise 0.7mm \hbox{${\scriptscriptstyle ++}$}}}

\begin{document}

\begin{center}
\textbf{\LARGE Dr Christian Rinderknecht}\\[5mm]
{\large Ingénieur compilation, langages dédiés, méthodes formelles}
\end{center}

\noindent
\begin{tabular}{@{}l@{\qquad\qquad\qquad\qquad}r@{}}
  Christian Rinderknecht
& \url{http://github.com/rinderknecht}\\
  Régi Posta utca 5 (6/1)
& \texttt{+36 70 311 6130}\\
  1052 Budapest
& \url{rinderknecht@free.fr}\\
  \textsc{Hongrie}
&
\end{tabular}

\medskip

\noindent{\small\url{http://www.linkedin.com/pub/christian-rinderknecht/47/421/193}}

\pagestyle{empty}

\bigskip
\noindent\textbf{\large Compétences clefs et Connaissances}
\begin{itemize*}

  \item Conception de langages et réalisation d'interprètes et de
    compilateurs.

  \item Ingénierie multidisciplinaire (génie logiciel, télécom,
    électronique).

  \item Ingénierie des protocoles et génération de tests à partir de
    modèles.

  \item Expérience internationale (France, Corée, Hongrie, Suède).

  \item Ex-universitaire et chercheur.

  \item Documentation technique et publications savantes.

  \item Bilingue Français/Espagnol et anglais courant (niveau C1 93\%).

\end{itemize*}

\bigskip

\noindent\textbf{\large Carrière}
\bigskip

\noindent\textbf{2023-} \ \textsf{Turnstiles Kft.} (Budapest,
Hungary)\\ \ \ \ \ \emph{Foundateur}\\
Expert indépendant en compilation, transformation de programmes,
systèmes à logique formelle, syntaxe et sémantique.

\bigskip

\noindent\textbf{2019-2023} \ \textsf{LIGO lang} (Paris,
France)\\ \ \ \ \ \emph{COO \& Ingénieur compilation}\\ Ingénieur
référent du développement du frontal du compilateur LIGO pour les
contrats intelligents sur la chaîne de blocs Tezos.

\bigskip

\noindent\textbf{2018-2019} \ \textsf{Nomadic Labs} (Paris,
  France)\\ \ \ \ \ \emph{Ingénieur compilation}\\ J'ai rejoint
  l'équipe créatrice de la chaîne de blocs Tezos.

\bigskip

\noindent\textbf{2017-2018} \ \textsf{GrAI Matter Labs} (Paris,
France)\\ \ \ \ \emph{Ingénieur compilation}\\ Conception d'un langage
pour décrire un nouveau type de réseau neuromorphique, et réalisation
d'un évaluateur et d'un traducteur vers OCaml.

\newpage

\noindent\textbf{2016} \ \textsf{Wolfram | MathCore} (Link\"oping,
Sweden)\\ \ \ \ \emph{Ingénieur compilation}\\ Conception et
réalisation d'analyseurs syntaxiques corrects et complets pour le
langage Modelica du produit Wolfram SystemModeler.

\bigskip

\noindent\textbf{2015-2016} \ \textsf{Numalis} (Montpellier,
France)\\ \ \ \ \emph{Ingénieur compilation}\\ Création d'outils
logiciels pour l'évaluation de la perte de précision numérique des
calculs en virgule flotante, par le biais de transformations
textuelles de programmes \cpp{} (Clang/LLVM).

\bigskip

\noindent\textbf{2014-2015} \ \textsf{Cortus} (Montpellier,
France)\\ \ \ \ \emph{Ingénieur compilation}\\ Maintenance et
réalisation d'un compilateur .NET (en C$^\sharp$ et OCaml).

\bigskip

\noindent\textbf{2001-2014} \ \textsf{\emph{Chercheur et
  universitaire}} (France, Corée, Hongrie)\\ (\textsf{École
  Supérieure d'Ingénieurs Léonard de Vinci}, \textsf{Konkuk
  University}, \textsf{Université E\"otv\"os Lor\'and}) R\&D dans le
domaine des compilateurs, de la vérification de protocoles, la
conception de langages dédiés etc. Enseignement de la théorie et de la
pratique de la programmation.

\bigskip

\noindent\textbf{2000} \ \textsf{PolySpace Tech.\@} (maintenant
MathWorks, Montbonnot, France)\\ \emph{Ingénieur R\&D}\\ Réalisation
d'un analyseur statique pour JavaCard, test automatique, études de cas
pour les prospects et support technique pour les commerciaux.

\bigskip

\noindent\textbf{1998-00} \ \textsf{Institut National des
  Télécommunications} (maintenant Télécom
SudParis)\\ \emph{Ingénieur R\&D} (Équipe Logiciels et
Réseaux)\\ Projets R\&D, génération de tests à partir de modèles pour
des services de télécommunication, réalisation d'outils pour le test
de protocoles.

\bigskip

\noindent\textbf{1997-98} \ \textsf{Alcatel-Alsthom CRC} (maintenant
Alcatel-Lucent R\&I, France)\\ \emph{Ingénieur} (Object
Architecture Unit)\\ Conception d'une métrique de qualité logicielle
pour un projet en \cpp{} (réseau).

\bigskip
\noindent\textbf{\large Formation}
\bigskip

\noindent\textbf{1993-98} \ \textsf{INRIA \& Université Pierre and
  Marie Curie} (France)\\ \emph{Doctorat en informatique (mention très
bien)}\\ Formalisation d'ASN.1, conception et réalisation d'un
analyseur de spécifications ASN.1. Preuve de correction des
\emph{Basic Encoding Rules} (BER). Participation au groupe de travail
ISO sur ASN.1 (Londres, \oldstylenums{1997}).


\newpage

%\bigskip
\noindent\textbf{\large Outils et langages formels}
\medskip
\begin{itemize*}

  \item \textit{Langages de programmation}: Java, OCaml, Erlang,
    C$^\sharp$, \cpp, XSLT, Ada, Standard ML, Prolog, Pascal.

  \item \textit{Technologies de formatage}: \LaTeX, XML, DTD, Markdown,
    JSON.

%  \item \textit{Protocol engineering}: ObjectGeode, Tau (Telelogic),
%    LOTOS, ASN.1, TTCN-3, MSC, SDL, specification-based test
%    generation, automata theory.

  \item \textit{Génie logiciel}: Génération de tests, conception et
    réalisation de compilateurs, analyse statique, transformation de
    programmes, méthodes formelles (formalisation, correction,
    complétude).

%  \item \textit{System administration}: Linux, OS~X.

%  \item \textit{Databases}: SQLite.

  \item \textit{Outils de programmation}: Emacs, GNU Make, dune, git,
    shell, sed, ocamllex, menhir etc.

  \item \textit{Logiciel libre}: \url{http://github.com/rinderknecht}

%%     Shell scripts for Linux system
%%     administration, a pretty\hyp{}printer for \TeX{} messages, a build
%%     system for OCaml, an OCaml library for dictionaries based on
%%     ternary search trees, a preprocessor for C$^\sharp$.

\end{itemize*}

\bigskip
\noindent\textbf{\large Publications et récompenses}
\bigskip
\begin{itemize*}

  \item 15 articles publiés dans des conférences et des revues
    internationales avec comités de lecture; 3~rapports techniques.

  \item \textit{Design and Analysis of Purely Functional Programs}
    (volume 15, \emph{Texts in Computing}, College Publications, UK,
    Nov~\oldstylenums{2012}, 660 pages). J'ai traduis mon livre pour
    le même éditeur: \textit{Conception et analyse des programmes
      purement fonctionnels} (volume 12, \emph{Cahiers de Logique et
    d'Épistémologie}, \oldstylenums{2012}).

  \item Un de mes articles mathématiques est la source d'une séquence
    d'entiers remarquable: \url{http://oeis.org/A261003}.

  \item J'ai reçu un chèque de Knuth pour avoir trouvé une erreur dans
    le volume~\oldstylenums{4} de \emph{The Art of Computer
    Programming}.

  \item J'ai traduit en français les \textit{Vingt poèmes
    d'amour et une chanson désespérée} de Pablo Neruda (Gallimard
    Poésie, Paris, \oldstylenums{1998}).

  \item J'ai composé et traduit en Espagnol une anthologie de poésie
    par Paul Valéry, \emph{Las granadas} (Ediciones Rilke, Madrid,
    \oldstylenums{2016}).

  \item J'ai contribué du vocabulaire bouddhique au \emph{Sanskrit
    Heritage} (dictionnaire Français-Sanscrit, \oldstylenums{1998}) de
    Gérard Huet\\ (\url{http://sanskrit.inria.fr/Heritage.pdf}).

  \item Je joue du violoncelle et j'écrie de la poésie.

\end{itemize*}

\bibliographystyle{unsrt}
%\bibliography{cv_rinderknecht}
\nocite{*}

\end{document}
