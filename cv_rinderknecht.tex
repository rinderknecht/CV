%%-*-latex-*-

\documentclass[a4paper,11pt]{article}

\usepackage[british]{babel} % British English
\usepackage[T1]{fontenc}    % Required for hyphenation
\usepackage{url}            % Typesets URLs
\usepackage{xspace}         % Adds space after macro expansions if needed
\usepackage{hyphenat}       % Hyphenation of compounded words
\usepackage{mdwlist}        % Tight vertical spacing of list items
\frenchspacing              % Settles the tricky business with \@

\newcommand{\cpp}{\mbox{C \hspace*{-2.5mm} \raise 0.7mm \hbox{${\scriptscriptstyle ++}$}}}

\begin{document}

\begin{center}
\textbf{\LARGE Dr Christian Rinderknecht}\\[5mm]
{\large Compiler Engineer, Domain-Specific Languages, Formal Methods}
\end{center}

\noindent
\begin{tabular}{@{}l@{\qquad\qquad\qquad\qquad}r@{}}
  Christian Rinderknecht
& \url{http://github.com/rinderknecht}\\
  Régi Posta utca 5 (6/1)
& \texttt{+36 70 311 6130}\\
  1052 Budapest
& \url{rinderknecht@free.fr}\\
  \textsc{Hungary}
&
\end{tabular}

\medskip

\noindent{\small\url{http://www.linkedin.com/pub/christian-rinderknecht/47/421/193}}

\pagestyle{empty}

\bigskip
\noindent\textbf{\large Key skills and Knowledge}
\begin{itemize*}

  \item Language Design and Interpreter/Compiler Construction.

  \item Multidisciplinary engineering (SE, telecom, electronics)

  \item Protocol Engineering and Model-based Test Generation

  \item International work experience (France, Korea, Hungary, Sweden)

  \item Ex-college professor and Researcher

  \item Technical Documentation and Scholarly Publications

  \item Bilingual French/Spanish and Fluent English (C1 level 93\%)

\end{itemize*}

\smallskip
\noindent\textbf{\large Employment History}
\bigskip

\noindent\textbf{2023-} \ \textsf{Turnstiles Kft.} (Budapest,
Hungary)\\ \ \ \ \ \emph{Founder}\\
Freelance expert in compilers, program transformations, logic sytems,
syntax and semantics.

\bigskip

\noindent\textbf{2019-2023} \ \textsf{LIGO lang} (Paris,
France)\\ \ \ \ \ \emph{COO \& Compiler Engineer}\\ Lead dev on the
front-end of the LIGO compiler for smart contracts on Tezos.

\bigskip

\noindent\textbf{2018-2019} \ \textsf{Nomadic Labs} (Paris,
  France)\\ \ \ \ \ \emph{Compiler Engineer}\\ Joined the core
  dev-team behind the Tezos cryptocurrency, where I worked on
  efficient, secure and self-governing blockchains with certified
  smart contracts, thanks to the OCaml programming language.

\bigskip

\noindent\textbf{2017-2018} \ \textsf{GrAI Matter Labs} (Paris,
France)\\ \ \ \ \emph{Compiler Engineer}\\ Design of a Domain-Specific
Language based on OCaml for describing a new kind of computational
neuromorphic spiking network, and implementation of a standalone
interpreter and a transpiler to OCaml.

\newpage

\noindent\textbf{2016} \ \textsf{Wolfram | MathCore}
(Link\"oping, Sweden)\\ \ \ \ \emph{Compiler Engineer}\\ Design and
implementation (using OCaml) of a correct and complete set of parsers
for the Modelica compiler of Wolfram SystemModeler, featuring a
precise, correct and complete set of syntax errors thanks to the
parser generator Menhir.

\bigskip

\noindent\textbf{2015-2016} \ \textsf{Numalis} (Montpellier,
France)\\ \ \ \ \emph{Compiler Engineer}\\ Development of tools in
\cpp{} and OCaml for assessing the loss of accuracy in floating-point
calculations, by means of source-to-source transformations (standalone
and based on Clang/LLVM) of \cpp{} code.

\bigskip

\noindent\textbf{2014-2015} \ \textsf{Cortus} (Montpellier,
France)\\ \ \ \ \emph{Compiler Engineer}\\ Maintenance and development
of a .NET compiler (in C$^\sharp$ and OCaml) for \textsf{Cortus}
microprocessors.

\bigskip

\noindent\textbf{2001-2014} \ \textsf{\emph{Researcher and University
    Professor}} (France, Korea, Hungary)\\ (\textsf{\'Ecole
  Sup\'erieure d'Ing\'enieurs L\'eonard de Vinci}, \textsf{Konkuk
  University}, \textsf{E\"otv\"os Lor\'and University}) R\&D on
compiler construction, protocol verification, domain\hyp{}specific language
design (Internet of Things), augmented reality, web-based framework
for e-learning. Teaching of programming.

\bigskip

\noindent\textbf{2000} \ \textsf{PolySpace Tech.\@} (now MathWorks,
Montbonnot, France)\\ \emph{R\&D Engineer}\\ Development of a static
analyser for JavaCard, automatic testing, reverse\hyp{}engineering and
maintenance, case studies and sales support.

\bigskip

\noindent\textbf{1998-00} \ \textsf{National Institute of
  Telecommunications} (now T\'el\'ecom SudParis)\\ \emph{R\&D
  Engineer} (Software for Networking Lab.)\\ R\&D projects,
specification-based test generation for telecommunication services,
development of tools for protocol testing.

\bigskip

\noindent\textbf{1997-98} \ \textsf{Alcatel-Alsthom CRC} (now
Alcatel-Lucent R\&I, France)\\ \emph{Case Engineer} (Object
Architecture Unit)\\ Design of a software quality analysis for a
\cpp{} project (networking).

\bigskip
\noindent\textbf{\large Education}
\bigskip

\noindent\textbf{1993-98} \ \textsf{INRIA \& Pierre and Marie Curie
  University} (France)\\ \emph{Ph.D.\@ in Informatics (cum
  laude)}\\ Formalisation of ASN.1, design and implementation of an
analyser for ASN.1. Soundness proof of the Basic Encoding
Rules (BER). Working group at ISO on ASN.1 (London, 1997).

\bigskip
\noindent\textbf{\large Tools and formal languages}
\medskip
\begin{itemize*}

  \item \textit{Programming languages}: Java, OCaml, Erlang,
    C$^\sharp$, \cpp, XSLT, Ada, Standard ML, Prolog, Pascal.

%  \item \textit{Markup technologies}: \LaTeX, XML, DTD.

%  \item \textit{Protocol engineering}: ObjectGeode, Tau (Telelogic),
%    LOTOS, ASN.1, TTCN-3, MSC, SDL, specification-based test
%    generation, automata theory.

  \item \textit{Software engineering}: Test generation, compiler
    construction, static analysis, program transformations, formal
    methods (specification, correctness) %, Docker Engine/Machine.

%  \item \textit{System administration}: Linux, OS~X.

%  \item \textit{Databases}: SQLite.

  \item \textit{Development tools}: Emacs, GNU Make, dune, git, shell
    scripting, scanning and parsing (sed, ocamllex, menhir) etc.

  \item \textit{Free Software}: \url{http://github.com/rinderknecht}

%%     Shell scripts for Linux system
%%     administration, a pretty\hyp{}printer for \TeX{} messages, a build
%%     system for OCaml, an OCaml library for dictionaries based on
%%     ternary search trees, a preprocessor for C$^\sharp$.

\end{itemize*}

\bigskip
\noindent\textbf{\large Publications and Honours}
\bigskip
\begin{itemize*}

  \item 15 papers in journals and conferences, 3~technical reports.

  \item \textit{Design and Analysis of Purely Functional Programs}
    (volume 15, \emph{Texts in Computing}, College Publications, UK,
    Nov~\oldstylenums{2012}, 660 pages). I translated my book into
    French for the same publisher: \textit{Conception et analyse des
      programmes purement fonctionnels} (volume 12, \emph{Cahiers de
    Logique et d'Épistémologie}, 2012).

  \item One of my mathematical articles is the source for the integer
    sequence \url{http://oeis.org/A261003}.

  \item I received a cheque from Knuth for finding an error in
    Volume~\oldstylenums{4} of \emph{The Art of Computer Programming}.

  \item Translator in French of the love poems \textit{Veintes poemas
    de amor y una canci\'on desesperada} of Pablo Neruda (Gallimard
    Po\'esie, Paris, \oldstylenums{1998}).

  \item Editor and translator in Spanish of an anthology of poetry by
    Paul Val\'ery, \emph{Las granadas} (Ediciones Rilke, Madrid,
    \oldstylenums{2016}).

  \item I contributed some Buddhist vocabulary to the Sanskrit
    Heritage (French-Sanskrit dictionary, \oldstylenums{1998}) of
    G\'{e}rard Huet\\ (\url{http://sanskrit.inria.fr/Heritage.pdf}).

  \item I am a cello player and I write poetry.

\end{itemize*}

\bibliographystyle{unsrt}
%\bibliography{cv_rinderknecht}
\nocite{*}

\end{document}
